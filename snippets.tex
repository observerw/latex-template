\documentclass[letterpaper]{article}
\usepackage{graphicx}
\usepackage{algorithm,algorithmicx,algpseudocode}

\begin{document}

\tableofcontents

\twocolumn

\section{Figures}

% 标准图片

\subsection{Standard Figures}

\begin{figure}
    \centering
    \includegraphics[width=0.2\textwidth]{assets/example.pdf}
    \caption{Example}
    \label{fig:standard}
\end{figure}

Figure~\ref{fig:standard} shows an example.

% 跨行图片(用于两列布局时)

\subsection{Wide Figures}

\begin{figure*}
    \centering
    \includegraphics[width=0.2\textwidth]{assets/example.pdf}
    \caption{Example}
    \label{fig:wide}
\end{figure*}

Figure~\ref{fig:wide} shows an example.

% 行内图片

% 使用 raisebox 将图片中轴线与文字对齐

\subsection{Inline Figures}

This is an inline image \begin{minipage}[b]{0.08\columnwidth}
    \centering
    \raisebox{-.05in}{\includegraphics[width=\linewidth]{assets/example.pdf}}
\end{minipage}.

% 图片矩阵

\subsection{Image Grid}

\begin{figure*}
    \centering
    \begin{minipage}[b]{0.3\textwidth}
        \centering
        \includegraphics[width=0.8\textwidth]{assets/example.pdf}
        \includegraphics[width=0.8\textwidth]{assets/example.pdf}
    \end{minipage}
    \begin{minipage}[b]{0.3\textwidth}
        \centering
        \includegraphics[width=0.8\textwidth]{assets/example.pdf}
        \includegraphics[width=0.8\textwidth]{assets/example.pdf}
    \end{minipage}
    \caption{Example}
    \label{fig:grid}
\end{figure*}

Figure~\ref{fig:grid} shows an example.

% 左一右二图片矩阵

% 使用 minipage 实现

\subsection{Left-One-Right-Two Image Grid}

\begin{figure*}
    \centering
    \begin{minipage}[b]{0.3\textwidth}
        \centering
        \includegraphics[width=0.8\textwidth]{assets/example.pdf}
    \end{minipage}
    \begin{minipage}[b]{0.3\textwidth}
        \centering
        \includegraphics[width=0.8\textwidth]{assets/example.pdf}
        \includegraphics[width=0.8\textwidth]{assets/example.pdf}
        \label{fig:grid2}
    \end{minipage}
    \caption{Example}
\end{figure*}

Figure~\ref{fig:grid2} shows an example.

\section{Tables}

% 标准表格

\subsection{Standard Tables}

\begin{table}
    \centering
    \begin{tabular}{|c|c|c|}
        \hline
        A & B & C \\
        \hline
        1 & 2 & 3 \\
        4 & 5 & 6 \\
        \hline
    \end{tabular}
    \caption{Example}
    \label{tab:standard}
\end{table}

Table~\ref{tab:standard} shows an example.

% 跨行表格

\subsection{Wide Tables}

\begin{table*}
    \centering
    \begin{tabular}{|c|c|c|}
        \hline
        A & B & C \\
        \hline
        1 & 2 & 3 \\
        4 & 5 & 6 \\
        \hline
    \end{tabular}
    \caption{Example}
    \label{tab:wide}

\end{table*}

Table~\ref{tab:wide} shows an example.

\section{Algorithms}

% 伪代码

\subsection{Pseudocode}

\begin{algorithm}
    \caption{Euclid’s algorithm}
    \label{algo:euclid}
    \begin{algorithmic}[1]
        \Procedure{Euclid}{$a,b$}
        \State $r\gets a\bmod b$
        \While{$r\not=0$}
        \State $a\gets b$
        \State $b\gets r$
        \State $r\gets a\bmod b$
        \EndWhile
        \State \textbf{return} $b$
        \EndProcedure
    \end{algorithmic}
\end{algorithm}

Algorithm~\ref{algo:euclid} shows an example.

\end{document}
